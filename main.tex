\documentclass{article}
\input{preamble.tex}

\addbibresource{mybib.bib}

\begin{document}

\title{Week 5 : Geometric Satake Part I - Affine Grassmannian}

\author{talk by Jonas Antor, notes by Ken Lee}
\date{Summer 2023}
\maketitle

Appendices contain details added after the talk.

\tableofcontents  

% Setup some notation : 
% \begin{itemize}
%   \item For $k$-algebra $A$,
%   $D^\wedge_A , D_A , D_A^\circ$ will denote
%   $\SPF A\bbrkt{t} , \SPEC A\bbrkt{t} , \SPEC A((t))$.
%   These are the relative formal disk, disk, punctured disk over $\SPEC A$.
%   \item 
% \end{itemize}

Let $G$ be a reductive group over $k = \bC$.
Recall the rough statement of geometric Satake.

\begin{prop}

  There is an equivalence of tensor abelian categories
  \[
    \SPH_G \simeq \REP_\bC G^L
  \]
\end{prop}

Here, the spherical Hecke category A.K.A. Satake category is 
the abelian category of $L^+G$-equivariant 
(regular holonomic) $D$-modules\footnote{
  or alternatively, under the Riemann--Hilbert correspondence,
  perverse sheaves
} on $\GR_G$.
\[
  \SPH_G := (D\MOD_{\text{rh}} \GR_G)^{L^+ G} \simeq
  (\PERV \GR_G)^{L^+ G}
\]
Today we focus on answering these two questions : 
\begin{enumerate}
  \item What is the algebro-geometric structure on $\GR_G$?
  \item How exactly do we work with equivariant $D$-modules on $\GR_G$?
\end{enumerate}

\section{The structure of the affine Grassmannian}

Recall from the talk in week 4 that
the affine Grassmannian is (equivalent to) the following set-valued functor 
\[
  \GR_G(A) = \set{(P , s) \st 
    \begin{matrix}
      \text{$G$-bundle $P$ on $D_A$} \\
      \text{$s : \res{\TRIV}{D_A^\circ} \simeq \res{P}{D_A^\circ}$}
    \end{matrix}
  } / \simeq
\]
where
\begin{itemize}
  \item $D_A = \SPEC A\bbrkt{t}$
  \item $D_A^\circ = \SPEC A((t))$
  \item $\res{\TRIV}{S} = S \times G$ trivial $G$-bundle
  \footnote{
    Recall $G$-bundles means fpqc $G$-torsor.
  }
\end{itemize}
Alternatively, we have 
\[
  \GR_G \simeq LG / L^+ G
\]
where the quotient is as fpqc sheaves.

We will see now that the affine Grassmannian is an \emph{ind-scheme}.
\begin{dfn}
  
  A functor $X \in \PSH \AFF$ is called an ind-scheme
  when it is a colimit of schemes along closed embeddings.
\end{dfn}

\begin{prop}
  
  $\GR_G$ is an ind-scheme.
  Moreover, we can write $\GR_G \simeq \COLIM_{N \in \bN} X_N$ where
  \begin{enumerate}
    \item $X_N$ is projective schemes over $k$
    \item transition maps are closed embeddings
    \item each $X_N$ is $L^+G$ stable
    \item The action of $L^+G$ on $X_N$ factors through some
    $L^M G$ for sufficiently large $M$.
  \end{enumerate}
\end{prop}
\begin{proof}
  ($G = \GL_n$ case)
  In this case we have equivalence between groupoids of 
  \begin{itemize}
    \item $\GL_n$-bundles on $D_A$
    \item rank $n$ vector bundles on $D_A$
    \item rank $n$ projective modules over $A\bbrkt{t}$
  \end{itemize}
  This allows us to make the following simplicification of 
  $\GR_{\GL_n}$ : 
  it is isomorphic to the functor
  \[
    A \mapsto \text{
    set of $A\bbrkt{t}$-lattices in $A((t))^n$
  }
  \]
  See \nameref{appendix:1}.
  
  Let $\La_0$ be the standard lattice $A\bbrkt{t}^n$ in $A((t))^n$.
  Then for any $A\bbrkt{t}$-lattice $\La$ one can find $N \leq 0$ such that 
  \[
    t^N \La_0 \subs \La \subs t^{-N}\La_0
  \]
  This condition defines a subfunctor $\GR_{\GL_n}^{(N)}$ for each $N$
  and we have \[
    \GR_{\GL_n} \simeq \COLIM_{N} \GR_{\GL_n}^{(N)}
  \]
  We choose $X_N = \GR_{\GL_n}^{(N)}$ for the result.

  Proof of (1) :
  \begin{itemize}
    \item Idea : each $\La \in \GR_{\GL_n}^{(N)}(A)$ gives
    a subspace 
    \[
      0 \to \La / t^N \La_0 \to t^{-N} \La_0 / t^N \La_0 \simeq A^{2nN}
    \]
    So $\La$ should give an $A$-point in the usual Grassmannian $\GR(2nN)$.
    \item Issue : $\GR(2nN)(A) =$ set of direct summands of $A^{2nN}$.
    \item Fix : we show that $\La / t^N \La_0$ has a complement in
    $t^{-N} \La_0 / t^N \La_0$.
    It suffices to show 
    \[
      \frac{t^{-N} \La_0 / t^N \La_0}{\La / t^N \La_0} \simeq
      \frac{t^{-N}\La_0}{\La}
    \]
    is projective over $A$.
    This fits in an SES : 
    \[
      0 \to \frac{t^{-N}\La_0}{\La} \to
      \frac{A((t))^n}{\La} \to \frac{A((t))^n}{t^{-N}\La_0} \to 0
    \]
    It suffices to show the right two modules are projective $A$-modules.
    The right-most module is free over $A$.
    For the middle module, we have 
    \[
      \frac{A((t))^n}{\La} = \frac{\bigcup_{0 \leq k} t^{-k} \La}{\La}
      \simeq \bigoplus_{0 \leq k} \frac{t^{-(k+1)}\La}{t^{-k}}
      \simeq \bigoplus_{0 \leq k} \frac{\La}{t \La}
    \]
    where the first isomorphism uses the fact that
    $\La / t \La$ is projective over $A$.

  \end{itemize}

  Proof of (2) :
  The above defines a map $\GR_{\GL_n}^{(N)} \to \GR(2nN)$
  which is injective,
  which we need to show is a closed embedding.
  This follows from the observation that subspaces of $t^{-N}\La_0 / t^N\La_0$ 
  of the form $\La / t^N \La_0$ are precisely those
  which are stable under the action of $t$.
  If we use $t$ again to denote the corresponding 
  $A$-linear endomorphism on $A^{2nN}$ 
  under $A^{2nN} \simeq t^{-N}\La_0 / t^N\La_0 \simeq A^{2nN}$,
  then we have
  \[
    \GR_{\GL_n}^{(N)}(A) \simeq \set{M \in \GR(2nN)(A) \st t M \subs M}
  \]
  The condition $tM \subs M$ is equivalent to $(M + t M) / M = 0$
  which defines a closed subscheme of $\SPEC A$.
  This shows $\GR_{\GL_n}^{(N)} \to \GR(2nN)$ is a closed embedding,
  and hence a projective scheme over $k$.

  Proof of (3) :
  Elements $g \in L^+\GL_n(A) = \GL_n(A\bbrkt{t})$ stabilise 
  the standard lattice $\La_0 = A\bbrkt{t}^n$.
  It follows that for any $\La \in \GR_{\GL_n}^{(N)}(A)$, 
  $g \La$ is still in $\GR_{\GL_n}^{(N)}(A)$.

  Proof of (4) :
  Recall that $L^M G(A) = G (A[t] / (t^{M+1}))$.
  The kernel of $L^+ \GL_n (A)\to L^M \GL_n (A)$ consists of
  elements of the form $1 + t^{M + 1} A$ with 
  $A \in M_{n \times n}(A\bbrkt{t})$.
  So the action of $L^+ G$ on $X_N$ factors through $L^M G$
  if and only if all $1 + t^{M + 1} A$ act trivially on 
  any points of $X_N$.
  We claim that $M + 1 = 2N$ works.
  It suffices to show $(1 + t^{2N}A) \La \subs \La$ for
  any $\La \in X_N$.
  This follows from 
  \[
    (1 + t^{2N} A) \La \subs \La + t^{2 N} A t^{-N} \La_0
    \subs \La + t^{N} \La_0 \subs \La
  \]

  (General reductive $G$)
  One can show that $\GR_G$ is ind-projective using the following argument :
  There exists $i : G \to \GL_n$ a closed embedding with
  affine homogeneous space $\GL_n / G$ of finite type.
  \footnote{
      All algebraic groups admit a closed embedding
      $G \to \GL_n$ for some $n$.
      \cite[Theorem 4.8]{milne}
      The fact that the quotient $\GL_n / G$ must be affine finite type
      is Matsushima's criterion.
  }
  To show $\GR_G$ is ind-projective,
  it suffices to show the induced functor $\GR_G \to \GR_{\GL_n}$
  is a closed embedding.
  See \nameref{appendix:2}.
  
\end{proof}


% \begin{enumerate}
%   \item $D_S := \SPEC \cO(S)\bbrkt{t}$. 
%   $G$-bundle on $D_S$ equivalent to monoidal, exact, faithful continuous
%   functor $\QCOH(\PT / G) \to \QCOH D_S$,
%   equivalently monoidal exact faithful $\PERF (\PT / G) \to \QCOH D_S$.
%   \item Monoidal preserves dualizable.
%   Hence equivalent to monoidal exact faithful $\PERF (\PT / G) \to \PERF D_S$.
%   \item For finite type $S$, $\COH D_S \simeq \COH D_S^\wedge$.
%   \cite[\href{https://stacks.math.columbia.edu/tag/087W}{Lemma 30.23.1}]{stacks-project}
%   One can check that $M \in \COH D_S$ corresponding to 
%   $(M_n)_n \in \COH D_S^\wedge$ is projective
%   iff each $M_n$ is projective.
%   So $G$-bundle on $D_S$ is equivalent to
%   monoidal exact faithful functor
%   $\PERF (\PT / G) \to \COH D_S^\wedge$ landing in
%   level-wise projective objects.

% \end{enumerate}

\section{Schubert varieties and cells}

This section is roughly based on \cite[Section 2.1]{Zhu-17}.

To define and study $L^+G$-equivariant $D$-modules on $\GR_G$,
we need to understand the $L^+G$-orbits (of $k$-valued points).
But we know that for a finite type scheme $X$, 
\[
  D\MOD(X) \map{\sim}{} D\MOD(X_\RED)
\]
so as far as $D$-modules are concerned,
we can work with $(\GR_G)_{\RED} = \COLIM_{n} (X_n)_\RED$ instead.
This is now an ind-variety\footnote{
  where variety means reduced, separated, finite type scheme over $k$
}
and hence we can work at the level of $k$-valued points.
We have\footnote{
  We showed in week 5 that $\GR_G = LG / L^+G$ where the quotient
  is as fpqc sheaves, so it does not
  immediately follow that $\GR_G(k) = LG(k) / L^+G(k)$.
  However, this is nonetheless true because $G$-bundles on $D_k$
  can be trivialised without passing through $D_S \to D_k$ 
  via an fpqc cover $S \to \SPEC k$.
}
\[
  (\GR_G)_\RED(k) = \GR_G(k) \simeq LG(k) / L^+G(k) = G(k((t))) / G(k\bbrkt{t})
\]

The $L^+G(k)$-orbits of $\GR_G(k)$ are given by the \emph{Cartan decomposition}.
Begin by choosing a maximal torus\footnote{
  automatically split because $k$ is algebraically closed
} and Borel $T \subs B \subs G$
and let $W := N(T) / T$ be the Weyl group.
Recall the coweight lattice of $T$ \[
  \mathbf{X}_\bullet(T) := \HOM(\bG_m , T)
\]
For any $F / k$ algebraically closed,
\begin{align*}
  \mathbf{X}_\bullet(T) &\to LG(F) \\
  \la &\mapsto t^\la := \la(t)
\end{align*}

\begin{prop}[Cartan decomposition]

  For $F / k$ algebraically closed,
  we have the isomorphism\footnote{
    Note that this isomorphism is functorial in $F$.
  }
  \[
    \mathbf{X}_\bullet(T)^+ \simeq 
    \mathbf{X}_\bullet(T) / W \simeq L^+G(F) \backslash LG(F) / L^+G(F)
  \]
\end{prop}

\begin{dfn}[Schubert cell, variety]
  
  For $\la \in \mathbf{X}_\bullet(T)^+$,
  the associated Schubert cell is defined as 
  \[
    \GR_G^\la := \text{ $L^+G(k)$-orbit of $t^\la$}
  \]
  and the associated Schubert variety is defined the Zariski closure 
  $\bar{\GR_G^\la}$.
\end{dfn}

\begin{prop}[Properties of Schubert cells and varieties]

  The following are true :
  \begin{enumerate}
    \item $\GR_G^\la \to \GR_G$ is locally closed variety
    \item $\bar{\GR_G^\la}$ is a projective variety
    \item For all $x \in \GR_G(k)$, $\mathrm{Stab}_{L^+G}(x)$ is connected
    \item $\dim \GR_G^\la = (2\rho , \la)$ 
    \item $\bar{\GR_G^\la} = \bigcup_{\mu \leq \la} \GR_G^\mu$
    \item The connected components of $\GR_G$ are 
    \[
      \bigcup_{\mu \in \om} \GR_G^\mu
      \text{ where $\om \in \mathbf{X}_\bullet(T) / \bZ\Phi^\vee$}
    \]
  \end{enumerate}
\end{prop}
\begin{proof}
  
  Recall that we could write $\GR_G = \COLIM_{n} X_n$ where
  the transition maps are closed embeddings and
  each $X_n$ is a $L^+G$-stable projective variety with
  $L^+G$ acting through a finite type quotient.
  For $\la \in \mathbf{X}_\bullet(T)^+$,
  $t^\la \in \GR_G(k)$ must lie in some $X_n$
  and hence $\GR_G^\la \subs X_n$ in that $X_n$.
  (1) and (2) now follow from \cite[Prop. 9.4]{milne}.

  (6) follows from (5).
  The proof of (3), (4), (5) was omitted in the talk
  but can be found at \cite[Prop. 2.1.5]{Zhu-16}.

\end{proof}

\section{Equivariant $D$-modules}

We recall some facts about $D$-modules on 
a smooth variety $X$ over $k = \bC$.

\begin{enumerate}
  \item We have the Riemann--Hilbert correspondence 
  \cite[Theorem 7.2.2]{HTT}
  \begin{cd}
    & {D^b_{rh}(D_X)} & {D^b_c(X^{\text{an}})} \\
    {D\MOD_{rh}(X)} & {{D^b_{rh}(D_X)}^{\heartsuit}} 
    & {D^b_c(X^{\text{an}})^{\heartsuit}} & {\PERV(X^\text{an})}
    \arrow["\simeq"{description}, draw=none, from=1-3, to=1-2]
    \arrow[from=2-2, to=1-2]
    \arrow[from=2-3, to=1-3]
    \arrow["\simeq"{description}, draw=none, from=2-2, to=2-3]
    \arrow["\simeq"{description}, draw=none, from=2-1, to=2-2]
    \arrow["{=:}"{description}, draw=none, from=2-4, to=2-3]
  \end{cd}
  The LHS is the derived category of complexes of $D$-modules
  with bounded cohomologies which are all regular holonomic.
  The RHS is the derived category of complexes of $\bC$-valued sheaves 
  with respect to the analytic topology on $X$,
  with bounded cohomologies which are all constructible.
  The $t$-structure on the LHS is the standard one and
  it identifies with the perverse $t$-structure on the right.
  For this study group, we are using the $D$-modules side,
  but Jonas is secretly using perverse sheaves so 
  to be safe, we will assume all $D$-modules are regular holonomic.
  \footnote{
    Holonomicity and regularity can be see as finiteness conditions
    \cite[Theorem 3.3.1, Definition 6.1]{HTT}.
  }
  \item There are six derived functors $f^!, f_! , f_* , f^*, \otimes, \HOM$
  on the $D^b(D_X)$.
  In general, none of these preserve the heart $D\MOD(\_)$.
  We list some special cases.
  \begin{itemize}
    \item If $i$ is a closed embedding, 
  then $i_* = i_!$ preserves $D\MOD(\_)$.
  \item If $j$ is an open embedding, 
  then $j^* = j^!$ preserves $D\MOD(\_)$.
  \item If $f$ is smooth of relative dimension $d$,
  then $f^\dagger := f^*[d] = f^![-d]$ preserves $D\MOD(\_)$
  \end{itemize}
\end{enumerate}

Recall from the previous talk that
given an algebraic group $H$ acting on $X$
we can consider the category $(D\MOD X)^H$ 
of (strongly) $H$-equivariant $D$-modules on $X$.
Given a morphism $H^\prime \to H$ of algebraic groups,
we have a restriction functor 
\[
  \mathrm{Res}^{H}_{H^\prime} : (D\MOD X)^{H} \to (D\MOD X)^{H^\prime}
\]
Concretely, 
$H^\prime \to H$ induces a map of the action groupoid of $H^\prime$ on $X$
to that of $H$, allowing restriction of $H$-equivariance
to $H^\prime$-equivariance.
For the example of $H^\prime = 1 \to H$,
we obtain the ``forgetful functor''
\[
  (D\MOD X)^H \to D\MOD X
\]
This is fully faithful when $H$ is connected.
\cite[Lem. A.1.2]{Zhu-16}

We have some further properties : 
\begin{enumerate}
  \item $i_* = i_! , j^* = j^! , f^\dagger$ still make sense 
  for (strongly) equivariant (regular holonomic) $D$-modules.
  \item
  Given $N \trianglelefteq H$ normal closed subgroup
  acting trivially on $X$,
  then 
  \[
    (D\MOD X)^{H / N} \map{\sim}{} (D\MOD X)^H
  \]
  \item Given $N \trianglelefteq H$ normal closed subgroup
  such that $N$ acts freely on $X$ and the fpqc quotient $X / N$
  is in fact an algebraic space,
  then 
  \[
    (D\MOD X / N)^{H / N} \map{\sim}{} (D\MOD X)^H
  \]

\end{enumerate}

From the above, we can show that
$H$-equivariant $D$-modules on homogeneous spaces of $H$
are very simple.
\begin{prop}

  Let $X$ be a homogeneous space for $H$ such that
  the stabilisers for $x \in X(k)$ are connected.
  \footnote{
    You only need one to be connected for all to be connected.
  }
  Then \[
    (D\MOD X)^H \simeq \VEC^{\text{f.d.}}_k
  \]
\end{prop}
\begin{proof}
  Pick $x \in X(k)$. Then $X \simeq H / H_x$.
  Then 
  \[
    (D\MOD X)^H \overset{(3)}{\simeq}
    (D\MOD H)^{H \times H_x} \overset{(3)}{\simeq}
    (D\MOD H/H)^{H_x} \simeq
    (D\MOD \PT)^{H_x} \overset{(2)}{\simeq}
    D\MOD \PT \simeq \VEC^{\text{f.d.}}_k
  \]
\end{proof}
This can be applied inductively using \emph{recollement} 
\cite[Exercises A.7.4 to A.7.7]{A21} to obtain the following :
\begin{prop}

  Let $H$ act on $X$ with finitely many orbits and
  suppose the stabiliser of all $x \in X(k)$ are connected.
  Then 
  \begin{align*}
    \text{$H$-orbits in $X$ } &\longleftrightarrow 
      \text{ Simples in $(D\MOD X)^H$} \\
    Hx &\mapsto \mathrm{IC}_{Hx}
  \end{align*}
  where $\mathrm{IC}_{Hx}$ is such that 
  \begin{itemize}
    \item $\supp \mathrm{IC}_{Hx} = \overline{Hx}$
    \item $\res{\mathrm{IC}_{Hx}}{Hx} \simeq \cO_{Hx}$ where
    restriction means $*$-pullback.
  \end{itemize}
\end{prop}
The definition of $\mathrm{IC}_{Hx}$ can be found in
\cite[Section 3.3]{A21}.

\section{$D$-modules on ind-varieties}

The situation is $X$ is a ind-variety with an action from
a pro-algebraic group $H$.
We require a presentation $X \simeq \COLIM_{i \in I} X_i$
where $X_i$ is $H$-stable and $H$ acts through a finite type quotient $H_i$.

Our desired situation is $X = \GR_{\GL_n}$,
$H = L^+\GL_n$, $X_i = \GR_G^{(i)}$ and $H_i = L^N\GL_n$ for large enough $N$.

We now define 
\[
  (D\MOD X_i)^H := (D\MOD X_i)^{H_i}
\]
Property (2) from the previous section ensures this is
independent of $H_i$ up to equivalence of categories.
We can now define
\[
  (D\MOD X)^H := \COLIM_{i} (D\MOD X_i)^H
\]
where
\begin{itemize}
  \item the objects are $(i, \cF_i)$ where $\cF_i \in (D\MOD X_i)^H$
  \item morphisms $(i , \cF_i) \to (j , \cF_j)$ are
  morphisms $\al_* \cF_i \to \be_* \cF_j$ where
  \[
    X_i \overset{\al}{\to} X_k \overset{\be}{\leftarrow} X_j
  \]
  This is well-defined because $*$-pushfoward is fully faithful along
  closed embeddings.
\end{itemize}
This category is independent of the presentation of $X$
up to equivalence.
Finally, we now have a definition of the spherical Hecke category.
\footnote{
  Thought to myself : 
  the collection of Schubert varieties indexed by $\mathbf{X}_\bullet(T)^+$
  should be a more canonical presentation of $\GR_G$
  than the one provided by $\GR_G \to \GR_{\GL_n}$.
  The issue is that this is really a presentation of $(\GR_G)_\RED$ not 
  $\GR_G$.
  But after speaking with Jonas,
  this seems to be a non-issue because 
  $D\MOD(X) \simeq D\MOD(X_\RED)$.
}
\begin{dfn}

  $\SPH_G := (D\MOD \GR_G)^{L^+ G}$
\end{dfn}

Given a closed subvariety $X$ of $\GR_G$ which is $L^+G$-stable,
it must be a union
\[
  X = \bigcup_{\la \in S} \GR_G^\la
\]
for some finite $S \subs \mathbf{X}_\bullet(T)^+$
because $X$ is finite dimensional and 
$\GR_G^\la$ have arbitrarily large dimension for larger $\la$.
It follows that $L^+G$ acts on $X$ through a finite type quotient
so $(D\MOD X)^{L^+G}$ makes sense.
Then we have a bijection
\[
  S \leftrightarrow \set{\text{ simples in $(D\MOD X)^{L^+G}$}}
\]
Since any $\cF \in (D\MOD \GR_G)^{L^+G}$ is supported on some
closed subvariety $X$ of $\GR_G$,
we deduce
\begin{prop}

  There is a bijection 
  \begin{align*}
    \mathbf{X}_\bullet(T)^+ &\longleftrightarrow 
      \set{\text{ simples in $(D\MOD \GR_G)^{L^+ G}$}} \\
    \la &\mapsto \mathrm{IC}_\la := \mathrm{IC}_{\GR_G^\la}
  \end{align*}
\end{prop}

TODO :
\begin{itemize}
  \item the technical issues with defining convolution
  \item the fix
  \item the sketch of proof that $\SPH_G$ is semi-simple
\end{itemize}

\section{Appendix : Affine Grassmannian for $\GL_n$}
\label{appendix:1}

\begin{lem}

  The affine Grassmannian for $\GL_n$ is isomorphic
  to the set-valued functor 
  \[
    A \mapsto \text{
    set of $A\bbrkt{t}$-lattices in $A((t))^n$
  }
  \]
where an $A\bbrkt{t}$-lattice in $A((t))^n$ is defined as
a $A\bbrkt{t}$ submodule $\La$ of $A((t))^n$ which is
\begin{enumerate}
  \item finitely generated projective over $A\bbrkt{t}$
  \item the inclusion $\La \subs A((t))^n$
  induces $\La \otimes_{A\bbrkt{t}} A((t)) \simeq A((t))^n$
\end{enumerate}
\begin{proof1}
  Let $\GR_{\GL_n}^\prime$ denote our target functor.
Let $\La$ be an $A$-family of lattices in $A((t))^n$.
Then $\La$ is a rank $n$ vector bundle on $D_A$.
The fact that the inclusion $\La \to A((t))^n$ induces
$\La \otimes_{A\bbrkt{t}} A((t)) \simeq A((t))^n$
supplies a trivialisation of $\res{\La}{D_S^\circ}$.
This defines $\GR_{\GL_n} \to \GR_{\GL_n}^\prime$.

Conversely, suppose we are given a rank $n$ vector bundle $M$ over $D_A$
and a trivialisation $s$ of $M$ over $D_A^\circ$.
Goal : give an $A\bbrkt{t}$-lattice $M$ in $A((t))^n$.
It suffices to show 
\[
  M \to M \otimes_{A\bbrkt{t}} A((t)) , m \mapsto m \otimes 1
\]
is an injective morphism of $A\bbrkt{t}$ modules
since the image of $M$ under $s : M \otimes_{A\bbrkt{t}} A((t)) \simeq A((t))^n$
gives an $A\bbrkt{t}$-lattice in $A((t))^n$.
We have a SES of $A\bbrkt{t}$ modules.
\[
  0 \to A\bbrkt{t} \to A((t)) \to Q \to 0  
\]
We reach our goal by applying $M \otimes_{A\bbrkt{t}} \_$ and
projectivity of $M$.
This defines $\GR_{\GL_n}^\prime \to \GR_{\GL_n}$.

We omit checking these are mutual inverses.
\end{proof1}
\end{lem}

\section{Appendix : Ind-projectivity of $\GR_G$ for general $G$}
\label{appendix:2}

  Goal : given an affine $S$ and a point $(P , s) : S \to \GR_{\GL_n}$,
  we need to show the right vertical morphism 
  \begin{cd}
    {\GR_G} & {?} \\
    {\GR_{\GL_n}} & S
    \arrow[from=1-1, to=2-1]
    \arrow[from=2-2, to=2-1, "{(P,s)}"]
    \arrow[from=1-2, to=1-1]
    \arrow[from=1-2, to=2-2]
    \arrow["\lrcorner"{anchor=center, pos=0.125, rotate=-90}, draw=none, from=1-2, to=2-1]
  \end{cd}
  is a closed embedding.
  By looking at the points of the fiber product in question,
  one sees that one needs to understand more about reductions
  of bundles,
  leading one to the following lemma :
  \begin{lem}
    
    Let $G_1 \to G_2$ be a closed subgroup inclusion of
    algebraic groups and
    and $X$ a functor acted on by $G_2$.
    Then we have a cartesian square : 
    \begin{cd}
      {X / G_1} & {\PT / G_1} \\
      {X/G_2} & {\PT / G_2}
      \arrow[from=1-1, to=1-2]
      \arrow[from=1-1, to=2-1]
      \arrow[from=2-1, to=2-2]
      \arrow[from=1-2, to=2-2]
      \arrow["\lrcorner"{anchor=center, pos=0.125}, draw=none, from=1-1, to=2-2]
    \end{cd}
    \begin{proof1}
      We only sketch comparison functors.
      \footnote{
        Note to self:
        actually I don't know where I used
        the fact that $G_1 \to G_2$ is a closed embedding.
      }
      A point of $X / G_1$ is a diagram 
      \begin{cd}
        P & X \\
        S
        \arrow["{G_1\text{-bundle}}"', from=1-1, to=2-1]
        \arrow["{G_1\text{-equiv}}", from=1-1, to=1-2]
      \end{cd}
      The map $X / G_1 \to \PT / G_1$ forgets the map $P \to X$.
      The map $X / G_1 \to X / G_2$ takes the above such diagrams to
      \begin{cd}
        {P \times^{G_1} G_2} & X \\
        S
        \arrow["{G_2\text{-bundle}}"', from=1-1, to=2-1]
        \arrow["{G_2\text{-equiv}}", from=1-1, to=1-2]
      \end{cd}
      The above defines a morphism 
      $X / G_1 \to (X / G_2) \times_{\PT / G_2} (\PT / G_1)$.

      We construct an inverse.
      A point of the fiber product is the data of
      an affine $S$, a $G_1$-bundle $P_1$ on $S$,
      a $G_2$-bundle $P_2$ on $S$,
      a $G_2$-equivariant map $P_2 \to X$ and
      an isomorphism of $G_2$-bundles $P_1 \times^{G_1} G_2 \simeq P_2$.
      Given this, we make the diagram 
      \begin{cd}
        {P_1} & {P_1 \times^{G_1} G_2} & {P_2} & X \\
        S
        \arrow["{G_2\text{-equiv}}", from=1-3, to=1-4]
        \arrow["\sim", from=1-2, to=1-3]
        \arrow["{G_1\text{-equiv}}", from=1-1, to=1-2]
        \arrow["{G_1\text{-bundle}}"', from=1-1, to=2-1]
      \end{cd}
    \end{proof1}
  \end{lem}
  Applying to our situation $G_1 = G , G_2 = \GL_n, X = P$
  shows that the groupoid of $G$-reductions of $P$
  is equivalent to the groupoid of sections of 
  $P / G \to D_S$.
  \footnote{
    The degenerate case of $G_1 = 1 , G_2 = G$
    says that a reduction of a $G$-bundle to a $1$-bundle
    is the same thing as a global section.
  }

  Now we can describe the points of the fiber product in question.
  First, see $s$ as a section of $\res{P}{D_S^\circ}$.
  Then given any affine $t : T \to S$ over $S$,
  then $s$ gives rise to a section $s_t$ of the pullback $P_t$
  to $D_T$ over $D_T^\circ$.
  \begin{cd}
    {(P_t)^\circ} & {P_t} \\
    {(P_t)^\circ / G} & {P_t / G} \\
    {D_T^\circ} & {D_T}
    \arrow[from=1-2, to=2-2]
    \arrow[from=1-1, to=2-1]
    \arrow[from=1-1, to=1-2]
    \arrow[from=3-1, to=3-2]
    \arrow[from=2-1, to=2-2]
    \arrow[from=2-1, to=3-1]
    \arrow[from=2-2, to=3-2]
    \arrow["{s_t}", bend left, from=3-1, to=1-1]
  \end{cd}
  Since $s_t$ trivialises $P_t^\circ$,
  it also provides a $G$-reduction of $P_t^\circ$.
  The question is : what are all the extensions of
  this $G$-reduction to all of $D_T$?
  Using the above lemma, 
  this is equivalent to asking for the
  groupoid of sections of $P_t / G$ extending $s_t$ 
  from $D_T^\circ$ to $D_T$.
  Note that the data of extending $s_t$ means
  this groupoid is discrete.
  Another way of seeing $P_t / G$ is
  as $P \times^{\GL_n} (\GL_n / G)$.
  By our assumption of $\GL_n / G$ being affine and finite type,
  it follows that $P_t / G$ is relatively affine and finite type over $D_T$.
  We can thus give a closed embedding 
  \begin{cd}
    {P_t / G} & {\bA^N_{D_T}} \\
    {D_T}
    \arrow[from=1-1, to=2-1]
    \arrow[from=1-2, to=2-1]
    \arrow["{\text{c.emb.}}", from=1-1, to=1-2]
  \end{cd}
  for some large $N$.
  The section $s_t$ of $P_t / G$ over $D_T^\circ$ can be seen as the data
  \[
    s_t = (s_t^1 , \dots , s_t^N) \in (\cO(T)((z)))^N
  \]
  One sees now that the locus of points of $D_T$
  admitting extensions of $s_t$ is given by
  the vanishing of the polar terms in the Laurent expansions of
  the coordinates of $s_t$.
  Furthermore, any extension is unique.
  We have thus shown that the fiber product in question
  is a close subscheme of $S$.

\section{Appendix : Schematic definition of Schubert cells and varieties}

The following is my understanding on how to define
the Schubert cells and varieties schematically
without working at the level of $k$-valued points.
As discussed with Jonas, 
since $D$-modules do not see the difference between
schemes and their reduction,
this is purely aesthetical.

\begin{dfn}

  Let $F / k$ be algebraically closed.
  Let $P_0, P_1$ be $G$-bundles on $D_F$ and 
  $\be : \res{P_0}{D_F^\circ} \to \res{P_1}{D_F^\circ}$.
  Given trivialisations $s_i$ of $P_i$ on $D_F$,
  one obtains an automorphism
  $s_1^{-1} \be s_0$ of $\res{\TRIV}{D_F^\circ}$,
  equivalently an element of $LG(F)$.
  Quotienting by the choice of trivialisations,
  we obtain a well-defined element 
  \[
    \INV(\be) \in L^+G(F) \backslash LG(F) / L^+G(F) \simeq 
    \mathbf{X}_\bullet(T)^+
  \]
  This is called the \emph{relative position of $\be$}.

  For $F$ not necessarily algebraically closed,
  one can choose an algebraic closure $\bar{F} / F$ and define
  $\INV(\be)$ by first base changing along $D_{\bar{F}} \to D_F$
  when do it over $D_{\bar{F}}$.
  The resulting element of $\mathbf{X}_\bullet(T)^+$
  is independent of the choice of $\bar{F}$ by the Cartan decomposition.

  Now let $\SPEC R$ be an affine scheme and
  $(P , s) \in \GR_G(R)$.
  For a topological point $x \in \SPEC R$,
  define 
  \[
    \INV_x s := 
    \text{ relative difference of }
    P_{D_{\ka(x)}}
     \map{s}{}
    \TRIV_{D_{\ka(x)}}
    \text{ over $D_{\ka(s)}^\circ$}
  \]
  For each $\mu \in \mathbf{X}_\bullet(T)^+$,
  we now define the Schubert variety 
  associated to $\mu$
  as a subfunctor of $\GR_G$ by the formula
  \[
    \GR_G^{\leq \mu} (R) := 
    \set{(P , s) : \forall\, x \in \SPEC R , \INV_x s \leq \mu}
  \]
  We also define the Schubert cell of $\mu$ as
  \[
    \GR_G^\mu := \GR_G^{\leq \mu} \setminus \bigcup_{\la < \mu} \GR_G^{\leq \la}
  \]

\end{dfn}

At this point, one should be able to either show
that $\GR_G^\la$ and $\GR_G^{\leq \la}$ have reductions giving
back the definition in the talk,
and maybe even show that these are reduced already.
But I have not had the time to think about this nor
have I found a reference.



\printbibliography

\end{document}